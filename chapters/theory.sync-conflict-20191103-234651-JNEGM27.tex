\section{Theory}

	\subsection{Reservoir Computing}
	
	%what is RC?
	Reservoir Computing is a subfield in Machine Learning. It is a newer and more general field derived from Echo-State-Networks. In Reservoir Computing the a "reservoir" is used in order to map an input of low dimension into a higher dimension. The reservoir can be a network as in an echo-state-network, but also a dynamic system with a set of certain properties. For example a bucket of water has been shown to be an effective reservoir in a pattern recognition setup.
	
	\subsubsection{Echo state networks}
	Echo state networks (ESNs) are Networks are a recurrent neural network model. 
	

	\subsection{Dynamics of a single Stuart-Landau-Oscillator}
	The Stuart-Landau oscillator is a dynamical system often used to model basic class 1 lasers. It consists of a single complex differential equation. 
	\begin{equation}	
		\dot{z}(t) = (\lambda +  i \omega + \gamma |z(t)|^2 ) \ z(t)
		\label{eq:stuartlandauequation}		
	\end{equation}

As can be seen in (eq.\ref{eq:stuartlandauequation}), the equation has a linear and a nonlinear term regarding the absolute value of $z$.


\subsection{Networks}

Vertices blabla \
Edges blabla. \

	\subsubsection{circulant Matrix}
    A circulant matrix has the same entries its row vectors, but with its entries rotated one element to the right relative to the previous row.
    
   
    
\subsection{virtual Nodes and multiplexing}
	
	
\subsection{Dynamics of rings of identical Stuart-Landau oscillators}

% different stuart landau scenarios - hopf bifurcation