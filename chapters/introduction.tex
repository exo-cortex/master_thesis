\chapter{one}

\section{Introduction}

	Reservoir Computing encompasses the field of machine learning in which a wide variety of systems (reservoirs) are used to contain information and perform calculations on it. Reservoir Computing is particularly interesting as computations can be performed by the physical system directly. The "classical" Turing-machine as a model is more like a virtual space created in our physical space. Within this virtual space everything is defined only in discrete units of "0" and "1". In order to create such a virtual space the usual unpredictability that inhabits the scales in which modern computer circuits exist in has to be tamed in order to create this virtual computing space. As the scales of modern transistors shrink they rapidly approach the scales in which quantum effects become problematic. The maintaining of this virtual space of $0$ and $1$ in which all our computations are performed is increasingly difficult. Modern CPU manufacturing has to take into account many error compensation algorithms in. Simultaneously manufacturing is becoming more challenging as well since the scales of modern transistors are so small the light sources needed for lithography are becoming rare.
	
	Reservoir computing is the circumventing of this virtual space of discrete values in order to perform computations directly in physical systems. A wide variety of systems can be used e.g. a literal bucket of water can act as a reservoir that performs computations \cite{FER03}. Albeit the most interesting applications lie in potential optical computers. RC offers a way of utilizing the highly complex dynamics of optical systems in order to perform computation on them. Optical Computers in the form of lasers appear to be an ideal application of RC, because of the timescales that laser dynamics. Optical reservoir computers already perform classification tasks on very timescales unmatched by modern silicon electronics.
	Another expected benefit would be the vastly lower power consumption.  
	

	In recent years reservoir computing has received a lot of attention as bridge between machine learning and physics. As the end of "Moore's Law" is slowly encroaching we are in the waning years of an age of staggering performance leaps in silicon electronic circuits. Reservoir Computing as a way of utilizing 
	
	\cite{SAN17a}

	\cite{LAR12} i fischer.


\subsection{blabla}

here comes the intro and all the important references...
